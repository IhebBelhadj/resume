\documentclass[10pt, letterpaper]{article}

% Packages:
\usepackage[
ignoreheadfoot, % set margins without considering header and footer
top=1 cm, % seperation between body and page edge from the top
bottom=1 cm, % seperation between body and page edge from the bottom
left=2 cm, % seperation between body and page edge from the left
right=2 cm, % seperation between body and page edge from the right
footskip=1.0 cm, % seperation between body and footer
% showframe % for debugging 
]{geometry} % for adjusting page geometry
\usepackage{titlesec} % for customizing section titles
\usepackage{tabularx} % for making tables with fixed width columns
\usepackage{array} % tabularx requires this
\usepackage[dvipsnames]{xcolor} % for coloring text
\definecolor{primaryColor}{RGB}{0, 0, 0} % define primary color
\usepackage{enumitem} % for customizing lists
\usepackage{fontawesome5} % for using icons
\usepackage{amsmath} % for math
\usepackage[
pdftitle={Iheb Belhadj's CV},
pdfauthor={Iheb Belhadj},
pdfcreator={Iheb Belhadj},
colorlinks=true,
urlcolor=primaryColor
]{hyperref} % for links, metadata and bookmarks
\usepackage[pscoord]{eso-pic} % for floating text on the page
\usepackage{calc} % for calculating lengths
\usepackage{bookmark} % for bookmarks
\usepackage{lastpage} % for getting the total number of pages
\usepackage{changepage} % for one column entries (adjustwidth environment)
\usepackage{paracol} % for two and three column entries
\usepackage{ifthen} % for conditional statements
\usepackage{needspace} % for avoiding page brake right after the section title
\usepackage{iftex} % check if engine is pdflatex, xetex or luatex

% Ensure that generate pdf is machine readable/ATS parsable:
\ifPDFTeX
  \input{glyphtounicode}
  \pdfgentounicode=1
  \usepackage[T1]{fontenc}
  \usepackage[utf8]{inputenc}
  \usepackage{lmodern}
\fi

\usepackage{charter}

% Some settings:
\raggedright
\AtBeginEnvironment{adjustwidth}{\partopsep0pt} % remove space before adjustwidth environment
\pagestyle{empty} % no header or footer
\setcounter{secnumdepth}{0} % no section numbering
\setlength{\parindent}{0pt} % no indentation
\setlength{\topskip}{0pt} % no top skip
\setlength{\columnsep}{0.15cm} % set column seperation
\pagenumbering{gobble} % no page numbering

\titleformat{\section}{\needspace{4\baselineskip}\bfseries\large}{}{0pt}{}[\vspace{1pt}\titlerule]

\titlespacing{\section}{
% left space:
  -1pt
}{
% top space:
  0.3 cm
}{
% bottom space:
  0.2 cm
} % section title spacing

\renewcommand\labelitemi{$\vcenter{\hbox{\small$\bullet$}}$} % custom bullet points
\newenvironment{highlights}{
  \begin{itemize}[
    topsep=0.10 cm,
    parsep=0.10 cm,
    partopsep=0pt,
    itemsep=0pt,
    leftmargin=0 cm + 10pt
    ]
  }{
\end{itemize}
} % new environment for highlights


\newenvironment{highlightsforbulletentries}{
  \begin{itemize}[
    topsep=0.10 cm,
    parsep=0.10 cm,
    partopsep=0pt,
    itemsep=0pt,
    leftmargin=10pt
    ]
  }{
\end{itemize}
} % new environment for highlights for bullet entries

\newenvironment{onecolentry}{
  \begin{adjustwidth}{
      0 cm + 0.00001 cm
    }{
      0 cm + 0.00001 cm
    }
  }{
  \end{adjustwidth}
} % new environment for one column entries

\newenvironment{twocolentry}[2][]{
  \onecolentry
  \def\secondColumn{#2}
  \setcolumnwidth{\fill, 4.5 cm}
  \begin{paracol}{2}
  }{
    \switchcolumn \raggedleft \secondColumn
  \end{paracol}
  \endonecolentry
} % new environment for two column entries

\newenvironment{threecolentry}[3][]{
  \onecolentry
  \def\thirdColumn{#3}
  \setcolumnwidth{, \fill, 4.5 cm}
  \begin{paracol}{3}
    {\raggedright #2} \switchcolumn
  }{
    \switchcolumn \raggedleft \thirdColumn
  \end{paracol}
  \endonecolentry
} % new environment for three column entries

\newenvironment{header}{
  \setlength{\topsep}{0pt}\par\kern\topsep\centering\linespread{1.5}
}{
  \par\kern\topsep
} % new environment for the header

\newcommand{\placelastupdatedtext}{% \placetextbox{<horizontal pos>}{<vertical pos>}{<stuff>}
  \AddToShipoutPictureFG*{% Add <stuff> to current page foreground
    \put(
    \LenToUnit{\paperwidth-2 cm-0 cm+0.05cm},
    \LenToUnit{\paperheight-1.0 cm}
    ){\vtop{{\null}\makebox[0pt][c]{
          \small\color{gray}\textit{Last updated in September 2024}\hspace{\widthof{Last updated in September 2024}}
    }}}%
  }%
}%

% save the original href command in a new command:
\let\hrefWithoutArrow\href

% new command for external links:


\begin{document}
\newcommand{\AND}{\unskip
  \cleaders\copy\ANDbox\hskip\wd\ANDbox
  \ignorespaces
}
\newsavebox\ANDbox
\sbox\ANDbox{$|$}

\begin{header}
  \fontsize{25 pt}{25 pt}\selectfont Iheb Belhadj

  \vspace{5 pt}

  \normalsize
  \kern 5.0 pt%
  \AND%
  \kern 5.0 pt%
  \AND%
  \mbox{\hrefWithoutArrow{mailto:iheb.belhadj.dev@gmail.com}{iheb.belhadj.dev@gmail.com}}%
  \kern 5.0 pt%
  \AND%
  \mbox{\hrefWithoutArrow{tel:+21695611874}{+216 95 611 874}}%

  \AND%
  \kern 5.0 pt%
  \mbox{\hrefWithoutArrow{https://yourwebsite.com/}{iheb.belhadj.com}}%
  \kern 5.0 pt%
  \AND%
  \kern 5.0 pt%
  \mbox{\hrefWithoutArrow{https://linkedin.com/in/iheb-belhaj/}{linkedin.com/in/iheb-belhaj}}%
  \kern 5.0 pt%
  \AND%
  \kern 5.0 pt%
  \mbox{\hrefWithoutArrow{https://github.com/IhebBelhadj}{github.com/IhebBelhadj}}%


\end{header}

\vspace{5 pt - 0.3 cm}


\section{Summary}


\begin{onecolentry}
Software engineer specializing in backend systems, data pipelines, and production-grade agentic systems. Experienced in architecting scalable backend services, building cloud-native applications, and developing agentic AI pipelines.
\end{onecolentry}


\section{Experience}




\begin{twocolentry}{
    Feb 2025 – Aug 2025
  }
\textbf{Software Engineering Intern}, Talan -- Tunis,TN\end{twocolentry}

\vspace{0.10 cm}
\begin{onecolentry}
  \begin{highlights}
\item Designed and implemented an AI-first analytics platform enabling business users to explore and analyze data through natural language queries.
  \item Integrated a multi-agent system with LangGraph, dynamically generating SQL queries and visualizations, with support for RAG pipelines and MCP integrations.
  \item Built a real-time change data capture (CDC) system, automatically refreshing charts when source data changed, improving analytics freshness.
  \item Developed an end-to-end data pipeline with MinIO, Delta Lake, Airflow, and dbt to streamline ingestion, transformation, and storage, while orchestrating ML forecasting jobs.
  \item Secured the platform using Keycloak IAM, implementing role- and group-based fine-grained access control for both application and data access using row-level security (RLS).
    \begin{highlightsforbulletentries}
    \item Technologies used: Python, FastAPI, Next.js, LangGraph, LangChain, MLflow, MinIO, Delta Lake, Airflow, dbt, PostgreSQL, Keycloak, Docker, Kubernetes, Nginx
    \end{highlightsforbulletentries}
\end{highlights}
    \end{highlights}
  \end{onecolentry}


  \vspace{0.2 cm}

  \begin{twocolentry}{
      Dec 2024 - Feb 2025
    }
  \textbf{Freelance Job}, Nexits  -- Paris, FR\end{twocolentry}

  \vspace{0.10 cm}
  \begin{onecolentry}
    \begin{highlights}
    \item Architected and developed CMS solution,
      leveraging Payload CMS and a custom-built, block-based page builder enabling nexits team to create new pages and edit content Nexits landing website with no code
    \item 
      Instantaneous Preview system that dynamically build components from declarative config from the CMS using Astro.js for minimal JS on build
    \item Dual rendering strategy with SSR for preview mode and SSG for production build
    \item Achieved 100 performance score on Google Lighthouse 95 accessibility and 100 seo score on production landing website using this architecture 

          \begin{highlightsforbulletentries}
    \item Technologies used: Astro.js, Payload CMS, React, TypeScript, Vercel, SQLite, Turborepo
    \end{highlightsforbulletentries}

    \end{highlights}

  \end{onecolentry}





  \begin{twocolentry}{
      Aug 2023 - Sep 2024
    }
  \textbf{Software engineer}, Nexits  -- Paris, FR\end{twocolentry}

  \vspace{0.10 cm}
  \begin{onecolentry}
    \begin{highlights}

    \item Implemented product state historization with full audit trail, enabling edits, rollbacks, and time-travel of product data.

    \item Contributed to the migration from a legacy system to a modern stack using Next.js, tRPC, and a monorepo architecture with Turborepo.

    \item Contributed in the deployment and management of monorepo services on AWS as well as CI/CD pipelines using GitHub Actions.

                \begin{highlightsforbulletentries}
    \item Technologies used: Next.js, React, TypeScript, Node.js, tRPC, PostgreSQL, AWS (EC2, S3, RDS), Docker, GitHub Actions, Turborepo
    \end{highlightsforbulletentries}


    \end{highlights}

  \end{onecolentry}


    \begin{twocolentry}{
      Oct 2023 - Nov 2023
    }
  \textbf{Freelance instructor}, ENET’Com Junior Entreprise  -- Sfax, TN\end{twocolentry}

  \vspace{0.10 cm}
  \begin{onecolentry}
    \begin{highlights}

    \item Conducted workshops over the weekends on web development using the MERN stack (MongoDB, Express.js, React, Node.js) teaching more than 30 junior club members.


    \end{highlights}

  \end{onecolentry}





  \begin{twocolentry}{
      June 2023 - Aug 2023
    }
  \textbf{Software Engineering Intern}, Nexits  -- Paris, FR\end{twocolentry}

  \vspace{0.10 cm}

  \section{Projects}


  \vspace{0.2 cm}

  \begin{twocolentry}{
      \href{https://github.com/sinaatalay/rendercv}{github.com/name/repo}
    }
  \textbf{Playpex}\end{twocolentry}

  \vspace{0.10 cm}
  \begin{onecolentry}
    \begin{highlights}
\item Built a streaming torrent client in Node.js with a backend wrapper leveraging Media Source Extensions API for seamless browser playback.
\item Designed a TV-like Angular interface optimized for remote and keyboard navigation, delivering a living-room experience.
\item Bundled torrent streaming backend and services into a unified Electron application for cross-platform desktop use.
\item Developed a companion Ionic mobile controller app, enabling real-time playback control via WebSockets.

                \begin{highlightsforbulletentries}
    \item Technologies used: Node.js, Angular, Ionic, Electron
    \end{highlightsforbulletentries}

    \end{highlights}
  \end{onecolentry}


  \section{Technologies}




  \begin{onecolentry}
    \textbf{Languages:} JavaScript, Typescript, Python, C\#, C, SQL, Java, PHP, Bash
  \end{onecolentry}

  \vspace{0.2 cm}

  \begin{onecolentry}
    \textbf{Backend dev} Spring, .NET, Express.js, Django, Flask, FastAPI
  \end{onecolentry}

  \vspace{0.2 cm}


  \begin{onecolentry}
    \textbf{Frontend dev} React, Next.js, Astro.js, Angular, Vue.js
  \end{onecolentry}


  \vspace{0.2 cm}

  \begin{onecolentry}
    \textbf{Databases:} MySQL, PostgreSQL, MongoDB, Redis, DuckDB, MinIO, RDS, S3
  \end{onecolentry}

  \vspace{0.2 cm}

  \begin{onecolentry}
    \textbf{Data eng: }  Airflow, Airbyte, dbt, Kafka, Spark
  \end{onecolentry}

  \vspace{0.2 cm}

  \begin{onecolentry}
    \textbf{Agentic AI:} LangChain, LangGraph, CrewAI, AutoGen
  \end{onecolentry}

  \vspace{0.2 cm}


  \begin{onecolentry}
    \textbf{Cloud:} AWS (EC2, S3, RDS, Lambda), GCP (Compute Engine, Cloud Storage, Cloud SQL)
  \end{onecolentry}

  \vspace{0.2 cm}


  \begin{onecolentry}
    \textbf{Infrastructure \& DevOps:} Docker, Kubernetes, Terraform, Ansible, Vagrant, Jenkins, GitHub Actions
  \end{onecolentry}






  \section{Education}




  \begin{twocolentry}{
      Sept 2022 – Sep 2025
    }
    \textbf{Masters in software engineering}, ESPRIT
  \end{twocolentry}

  \vspace{0.10 cm}

  \begin{twocolentry}{
      Sept 2020 – Juin 2022
    }
    \textbf{Bachelor in applied math and physics}, IPEIM
  \end{twocolentry}




  \end{document}
