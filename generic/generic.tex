\documentclass[10pt, letterpaper]{article}


% Packages:
\usepackage[
ignoreheadfoot, % set margins without considering header and footer
top=.5 cm, % seperation between body and page edge from the top
bottom=.5 cm, % seperation between body and page edge from the bottom
left=1 cm, % seperation between body and page edge from the left
right=1 cm, % seperation between body and page edge from the right
footskip=1.0 cm, % seperation between body and footer
% showframe % for debugging 
]{geometry} % for adjusting page geometry
\usepackage{titlesec} % for customizing section titles
\usepackage{tabularx} % for making tables with fixed width columns
\usepackage{array} % tabularx requires this
\usepackage[dvipsnames]{xcolor} % for coloring text
\definecolor{primaryColor}{RGB}{0, 0, 0} % define primary color
\usepackage{enumitem} % for customizing lists
\usepackage{fontawesome5} % for using icons
\usepackage{amsmath} % for math
\usepackage[
pdftitle={Iheb Belhadj's CV},
pdfauthor={Iheb Belhadj},
pdfcreator={Iheb Belhadj},
colorlinks=true,
urlcolor=primaryColor
]{hyperref} % for links, metadata and bookmarks
\usepackage[pscoord]{eso-pic} % for floating text on the page
\usepackage{calc} % for calculating lengths
\usepackage{bookmark} % for bookmarks
\usepackage{lastpage} % for getting the total number of pages
\usepackage{changepage} % for one column entries (adjustwidth environment)
\usepackage{paracol} % for two and three column entries
\usepackage{ifthen} % for conditional statements
\usepackage{needspace} % for avoiding page brake right after the section title
\usepackage{iftex} % check if engine is pdflatex, xetex or luatex

% Ensure that generate pdf is machine readable/ATS parsable:
\ifPDFTeX
  \input{glyphtounicode}
  \pdfgentounicode=1
  \usepackage[T1]{fontenc}
  \usepackage[utf8]{inputenc}
  \usepackage{lmodern}
\fi

\usepackage{charter}

% Some settings:
\raggedright
\AtBeginEnvironment{adjustwidth}{\partopsep0pt} % remove space before adjustwidth environment
\pagestyle{empty} % no header or footer
\setcounter{secnumdepth}{0} % no section numbering
\setlength{\parindent}{0pt} % no indentation
\setlength{\topskip}{0pt} % no top skip
\setlength{\columnsep}{0.15cm} % set column seperation
\pagenumbering{gobble} % no page numbering

\titleformat{\section}{\needspace{4\baselineskip}\bfseries\large}{}{0pt}{}[\vspace{1pt}\titlerule]

\titlespacing{\section}{
% left space:
  -1pt
}{
% top space:
  0.3 cm
}{
% bottom space:
  0.2 cm
} % section title spacing

\renewcommand\labelitemi{\vcenter{\hbox{\small$\bullet$}}} % custom bullet points
\newenvironment{highlights}{
  \begin{itemize}[
    topsep=0.10 cm,
    parsep=0.10 cm,
    partopsep=0pt,
    itemsep=0pt,
    leftmargin=0 cm + 10pt
    ]
  }{
\end{itemize}
}


\newenvironment{highlightsforbulletentries}{
  \begin{itemize}[
    topsep=0 cm,
    parsep=0.10 cm,
    partopsep=0pt,
    itemsep=0pt,
    leftmargin=10pt
    ]
  }{
\end{itemize}
} % new environment for highlights for bullet entries

\newenvironment{onecolentry}{
  \begin{adjustwidth}{
      0 cm + 0.00001 cm
    }{
      0 cm + 0.00001 cm
    }
  }{
  \end{adjustwidth}
} % new environment for one column entries

\newenvironment{twocolentry}[2][]{
  \onecolentry
  \def\secondColumn{#2}
  \setcolumnwidth{\fill, 4.5 cm}
  \begin{paracol}{2}
  }{
    \switchcolumn \raggedleft \secondColumn
  \end{paracol}
  \end{onecolentry}
} % new environment for two column entries

\newenvironment{threecolentry}[3][]{
  \onecolentry
  \def\thirdColumn{#3}
  \setcolumnwidth{, \fill, 4.5 cm}
  \begin{paracol}{3}
    {\raggedright #2} \switchcolumn
  }{
    \switchcolumn \raggedleft \thirdColumn
  \end{paracol}
  \end{onecolentry}
} % new environment for three column entries

\newenvironment{header}{
  \setlength{\topsep}{0pt}\par\kern\topsep\centering\linespread{1}
}{
  \par\kern\topsep
} % new environment for the header

\newcommand{\placelastupdatedtext}{% \placetextbox{<horizontal pos>}{<vertical pos>}{<stuff>}
  \AddToShipoutPictureFG*{
    \put(
    \LenToUnit{\paperwidth-2 cm-0 cm+0.05cm},
    \LenToUnit{\paperheight-1.0 cm}
    ){\vtop{{\null}\makebox[0pt][c]{
          \small\color{gray}\textit{Last updated in September 2024}\hspace{\widthof{Last updated in September 2024}}
    }}}
  }
}

% save the original href command in a new command:
\let\hrefWithoutArrow\href

% new command for external links:


\begin{document}
\newcommand{\AND}{\unskip
  \cleaders\copy\ANDbox\hskip\wd\ANDbox
  \ignorespaces
}
\newsavebox\ANDbox
\sbox\ANDbox{$|$}

\begin{header}
  \fontsize{25 pt}{25 pt}\selectfont Iheb Belhadj


  \normalsize
  \kern 5.0 pt%
  \AND%
  \kern 5.0 pt%
  \AND%
  \mbox{\hrefWithoutArrow{mailto:iheb.belhadj.dev@gmail.com}{iheb.belhadj.dev@gmail.com}}%
  \kern 5.0 pt%
  \AND%
  \mbox{\hrefWithoutArrow{tel:+21695611874}{+216 95 611 874}}%

  \AND%
  \kern 5.0 pt%
  \mbox{\hrefWithoutArrow{https://ihebbelhadj.vercel.app/}{ihebbelhadj.vercel.app}}%
  \kern 5.0 pt%
  \AND%
  \kern 5.0 pt%
  \mbox{\hrefWithoutArrow{https://linkedin.com/in/iheb-belhaj/}{linkedin.com/in/iheb-belhaj}}%
  \kern 5.0 pt%
  \AND%
  \kern 5.0 pt%
  \mbox{\hrefWithoutArrow{https://github.com/IhebBelhadj}{github.com/IhebBelhadj}}%


\end{header}

\vspace{5 pt - 0.3 cm}


\section{Summary}


\begin{onecolentry}
Software engineer specializing in backend systems, data pipelines, and agentic AI. Experienced in building cloud-native applications, multi-agent systems, and scalable infrastructure for data-driven solutions.
\end{onecolentry}

% Technologies
\section{Technologies}

\begin{onecolentry}
\textbf{Languages:} TypeScript, Python, JavaScript, Java, C\#, C, SQL, PHP, Bash
\end{onecolentry}

\begin{onecolentry}
\textbf{Web/Fullstack:} Next.js, React, Astro.js, Angular, Vue.js, Express.js, Spring, Flask, FastAPI
\end{onecolentry}

\begin{onecolentry}
\textbf{Databases:} PostgreSQL, MySQL, MongoDB, Redis, MinIO, AWS RDS, S3
\end{onecolentry}

\begin{onecolentry}
\textbf{Data Eng \& AI:} Airflow, Airbyte, dbt, Kafka, Spark, LangChain, LangGraph, MLflow
\end{onecolentry}

\begin{onecolentry}
\textbf{Infrastructure:} Docker, Kubernetes, Ansible, Vagrant, Jenkins, GitHub Actions, AWS
\end{onecolentry}

\section{Experience}




\begin{twocolentry}{
    Feb 2025 – Aug 2025
  }
\textbf{Software Engineering Intern}, Talan -- Tunis,TN\end{twocolentry}

\vspace{0.10 cm}
\begin{onecolentry}
  \begin{highlights}

  \item Integrated a multi-agent system with LangGraph for dynamic SQL and chart generation, enabling no-code data exploration.
  \item Built a real-time CDC pipeline refreshing charts live, enabling instant monitoring.
  \item Developed an end-to-end data pipeline with MinIO, Delta Lake, Airflow, and dbt to manage ingestion, transformation, and ML forecasting.
  \item Secured the platform with Keycloak IAM using role- and group-based access control.
    \begin{highlightsforbulletentries}
    \item Technologies used: Python, FastAPI, Next.js, LangGraph, LangChain, MLflow, MinIO, Delta Lake, Airflow, dbt, PostgreSQL, Keycloak, Docker, Nginx
    \end{highlightsforbulletentries}
\end{highlights}
    \end{highlights}
  \end{onecolentry}


  \vspace{0.2 cm}

  \begin{twocolentry}{
      Dec 2024 - Jan 2025
    }
  \textbf{Freelance Job}, Nexits  -- Paris, FR\end{twocolentry}

  \vspace{0.10 cm}
  \begin{onecolentry}
    \begin{highlights}
  \item Architected a CMS solution with Payload CMS and a custom block-based page builder, enabling no-code content updates for the landing site.
  \item Built an instantaneous preview system using Astro.js for dynamic component rendering with minimal JS.
  \item Designed a dual rendering strategy (SSR for preview, SSG for production), achieving 100 performance, 95 accessibility, and 100 SEO score on Lighthouse.

          \begin{highlightsforbulletentries}
    \item Technologies used: Astro.js, Payload CMS, React, TypeScript, Vercel, SQLite, Turborepo
    \end{highlightsforbulletentries}

    \end{highlights}

  \end{onecolentry}




  \begin{twocolentry}{
      Aug 2023 - Sep 2024
    }
  \textbf{Software engineer}, Nexits  -- Paris, FR\end{twocolentry}

  \vspace{0.10 cm}
  \begin{onecolentry}
    \begin{highlights}

  \item Implemented product state historization with audit trail, enabling edits, rollbacks, and time-travel of product data.
  \item Contributed to migration from legacy stack to Next.js, tRPC, and Turborepo monorepo with AWS deployment.

                \begin{highlightsforbulletentries}
    \item Technologies used: Next.js, React, TypeScript, Node.js, tRPC, PostgreSQL, AWS (EC2, ECR, S3, RDS), Docker, GitHub Actions, Turborepo
    \end{highlightsforbulletentries}


    \end{highlights}

  \end{onecolentry}


    \begin{twocolentry}{
      Oct 2023 - Nov 2023
    }
  \textbf{Freelance instructor (Weekends)}, ENET’Com Junior Entreprise  -- Sfax, TN\end{twocolentry}

  \vspace{0.10 cm}
  \begin{onecolentry}
    \begin{highlights}

    \item Conducted workshops over the weekends on web development using the MERN stack (MongoDB, Express.js, React, Node.js) teaching more than 30 junior club members.


    \end{highlights}

  \end{onecolentry}




  \begin{twocolentry}{
      June 2023 - Aug 2023
    }
  \textbf{Software Engineering Intern}, Nexits  -- Paris, FR\end{twocolentry}

  % \vspace{0.10 cm}



% Projects
\section{Projects}

\vspace{0.2 cm}
\begin{onecolentry}
  \begin{highlights}
  \item \textbf{Portfolio:} Built on Next.js with MDX-based content management, lazy loading, and server components. Integrated a LangChain agent that answers questions about my experience. Achieved 100 performance and best practices scores on Lighthouse.
    \begin{highlightsforbulletentries}
    \item Technologies used: Next.js, LangChain, Vercel
    \end{highlightsforbulletentries}
  \end{highlights}
\end{onecolentry}

\begin{onecolentry}
  \begin{highlights}
  \item \textbf{OpsForge:} A self-contained local DevOps environment simulating production using Vagrant and Ansible. Includes a private Docker registry, Jenkins CI/CD, monitoring stack with Prometheus/Grafana, and SonarQube.
    \begin{highlightsforbulletentries}
    \item Technologies used: Ansible, Vagrant, Docker, Jenkins, Prometheus, Grafana, SonarQube, Bash
    \end{highlightsforbulletentries}
  \end{highlights}
\end{onecolentry}
\begin{onecolentry}
  \begin{highlights}
  \item \textbf{Playpex:} Torrent streamer in Node.js with backend wrapper using MSE API for seamless browser playback. Angular frontend, Electron desktop app to bundle the engine and the frontend, and Ionic mobile controller.
  \end{highlights}
\end{onecolentry}


  \section{Education}





  \begin{twocolentry}{
      Sept 2022 – Sep 2025
    }
    \textbf{Software engineering}, ESPRIT
  \end{twocolentry}

  \vspace{0.10 cm}

  \begin{twocolentry}{
      Sept 2020 – Juin 2022
    }
    \textbf{Pre-engineering school (Math-Physics: Rank 626 )}, IPEIM
  \end{twocolentry}




  \end{document}
